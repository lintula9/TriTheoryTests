% Options for packages loaded elsewhere
\PassOptionsToPackage{unicode}{hyperref}
\PassOptionsToPackage{hyphens}{url}
\PassOptionsToPackage{dvipsnames,svgnames,x11names}{xcolor}
%
\documentclass[
  letterpaper,
  DIV=11,
  numbers=noendperiod]{scrartcl}

\usepackage{amsmath,amssymb}
\usepackage{iftex}
\ifPDFTeX
  \usepackage[T1]{fontenc}
  \usepackage[utf8]{inputenc}
  \usepackage{textcomp} % provide euro and other symbols
\else % if luatex or xetex
  \usepackage{unicode-math}
  \defaultfontfeatures{Scale=MatchLowercase}
  \defaultfontfeatures[\rmfamily]{Ligatures=TeX,Scale=1}
\fi
\usepackage{lmodern}
\ifPDFTeX\else  
    % xetex/luatex font selection
\fi
% Use upquote if available, for straight quotes in verbatim environments
\IfFileExists{upquote.sty}{\usepackage{upquote}}{}
\IfFileExists{microtype.sty}{% use microtype if available
  \usepackage[]{microtype}
  \UseMicrotypeSet[protrusion]{basicmath} % disable protrusion for tt fonts
}{}
\makeatletter
\@ifundefined{KOMAClassName}{% if non-KOMA class
  \IfFileExists{parskip.sty}{%
    \usepackage{parskip}
  }{% else
    \setlength{\parindent}{0pt}
    \setlength{\parskip}{6pt plus 2pt minus 1pt}}
}{% if KOMA class
  \KOMAoptions{parskip=half}}
\makeatother
\usepackage{xcolor}
\setlength{\emergencystretch}{3em} % prevent overfull lines
\setcounter{secnumdepth}{-\maxdimen} % remove section numbering
% Make \paragraph and \subparagraph free-standing
\ifx\paragraph\undefined\else
  \let\oldparagraph\paragraph
  \renewcommand{\paragraph}[1]{\oldparagraph{#1}\mbox{}}
\fi
\ifx\subparagraph\undefined\else
  \let\oldsubparagraph\subparagraph
  \renewcommand{\subparagraph}[1]{\oldsubparagraph{#1}\mbox{}}
\fi


\providecommand{\tightlist}{%
  \setlength{\itemsep}{0pt}\setlength{\parskip}{0pt}}\usepackage{longtable,booktabs,array}
\usepackage{calc} % for calculating minipage widths
% Correct order of tables after \paragraph or \subparagraph
\usepackage{etoolbox}
\makeatletter
\patchcmd\longtable{\par}{\if@noskipsec\mbox{}\fi\par}{}{}
\makeatother
% Allow footnotes in longtable head/foot
\IfFileExists{footnotehyper.sty}{\usepackage{footnotehyper}}{\usepackage{footnote}}
\makesavenoteenv{longtable}
\usepackage{graphicx}
\makeatletter
\def\maxwidth{\ifdim\Gin@nat@width>\linewidth\linewidth\else\Gin@nat@width\fi}
\def\maxheight{\ifdim\Gin@nat@height>\textheight\textheight\else\Gin@nat@height\fi}
\makeatother
% Scale images if necessary, so that they will not overflow the page
% margins by default, and it is still possible to overwrite the defaults
% using explicit options in \includegraphics[width, height, ...]{}
\setkeys{Gin}{width=\maxwidth,height=\maxheight,keepaspectratio}
% Set default figure placement to htbp
\makeatletter
\def\fps@figure{htbp}
\makeatother

\KOMAoption{captions}{tableheading}
\makeatletter
\makeatother
\makeatletter
\makeatother
\makeatletter
\@ifpackageloaded{caption}{}{\usepackage{caption}}
\AtBeginDocument{%
\ifdefined\contentsname
  \renewcommand*\contentsname{Table of contents}
\else
  \newcommand\contentsname{Table of contents}
\fi
\ifdefined\listfigurename
  \renewcommand*\listfigurename{List of Figures}
\else
  \newcommand\listfigurename{List of Figures}
\fi
\ifdefined\listtablename
  \renewcommand*\listtablename{List of Tables}
\else
  \newcommand\listtablename{List of Tables}
\fi
\ifdefined\figurename
  \renewcommand*\figurename{Figure}
\else
  \newcommand\figurename{Figure}
\fi
\ifdefined\tablename
  \renewcommand*\tablename{Table}
\else
  \newcommand\tablename{Table}
\fi
}
\@ifpackageloaded{float}{}{\usepackage{float}}
\floatstyle{ruled}
\@ifundefined{c@chapter}{\newfloat{codelisting}{h}{lop}}{\newfloat{codelisting}{h}{lop}[chapter]}
\floatname{codelisting}{Listing}
\newcommand*\listoflistings{\listof{codelisting}{List of Listings}}
\makeatother
\makeatletter
\@ifpackageloaded{caption}{}{\usepackage{caption}}
\@ifpackageloaded{subcaption}{}{\usepackage{subcaption}}
\makeatother
\makeatletter
\@ifpackageloaded{tcolorbox}{}{\usepackage[skins,breakable]{tcolorbox}}
\makeatother
\makeatletter
\@ifundefined{shadecolor}{\definecolor{shadecolor}{rgb}{.97, .97, .97}}
\makeatother
\makeatletter
\makeatother
\makeatletter
\makeatother
\ifLuaTeX
  \usepackage{selnolig}  % disable illegal ligatures
\fi
\IfFileExists{bookmark.sty}{\usepackage{bookmark}}{\usepackage{hyperref}}
\IfFileExists{xurl.sty}{\usepackage{xurl}}{} % add URL line breaks if available
\urlstyle{same} % disable monospaced font for URLs
\hypersetup{
  pdftitle={Theoretical study Quarto -document},
  colorlinks=true,
  linkcolor={blue},
  filecolor={Maroon},
  citecolor={Blue},
  urlcolor={Blue},
  pdfcreator={LaTeX via pandoc}}

\title{Theoretical study Quarto -document}
\author{}
\date{2026-01-28}

\begin{document}
\maketitle
\ifdefined\Shaded\renewenvironment{Shaded}{\begin{tcolorbox}[boxrule=0pt, enhanced, sharp corners, borderline west={3pt}{0pt}{shadecolor}, frame hidden, breakable, interior hidden]}{\end{tcolorbox}}\fi

\hypertarget{to-do-22.1.}{%
\subsubsection{To do 22.1.}\label{to-do-22.1.}}

\begin{itemize}
\item
  Add/integrate previous Rmd file of theoretical result as extension to
  this one (exact place TBD 30.1.).
\item
  Integrating Tom's notes (26.1 onwards)
\end{itemize}

\hypertarget{strict-sense-stationary-var1-and-strict-longitudinal-measurement-invariance}{%
\subsubsection{Strict sense stationary VAR(1) and strict longitudinal
measurement
invariance}\label{strict-sense-stationary-var1-and-strict-longitudinal-measurement-invariance}}

In the following VAR(1) process and strict longitudinal measurement
invariance (s-LMI) are compared to each other by the covariance
structure that VAR(1) imposes and how an s-LMI model fits to this (here
equivalently `is compatible' with) covariance. s-LMI makes sense as a
theoretical model since it captures the simplest scenario of a true
common factor model where only the common factor itself can change.
VAR(1) is also a simple, if not the simplest, vector autoregression
symptom-network model. CT-VAR and VAR can be linked through a
transformation when fixed time intervals are used, or separate study for
CT-VAR can be done.

Especially noteworthy results from mathematical analysis could be
possible contradictions that arise and how these contradictions guide
empirical simulations and tell us the reason why s-LMI might not be
compatible with VAR(1) generated data in the first place. Alternatively,
if some VAR(1) models can generate s-LMI compatible data, what
constraints are necessary for the VAR(1) process to produce it?

First VAR(1) imposed covariance is derived. Then s-LMI imposed
covariance. Then we move on to inspect how they compare by equating them
together to observe possible contradictions or restrictions. We begin
with the simplest and possibly most common scenario where two subsequent
measurement time points (from hereon simply time points) of symptoms are
observed.

\hypertarget{var1-covariance-structure-at-two-subsequent-time-points}{%
\paragraph{VAR(1) covariance structure at two subsequent time
points}\label{var1-covariance-structure-at-two-subsequent-time-points}}

The VAR(1) model is defined in matrix format as
\(X_{t}=C+AX_{t-1}+\Gamma_t\), where \(\Gamma_t\) is independent error
column vector with \(E[\Gamma_t]=0\), \(C\) is a constant assumed zero.
Also assume centered \(X\), \(E[X_t]=0\), in our case. Centering makes
covariance calculations easier as the products of expected values can be
mostly ignored (they become 0). \(A\) is \(K \times K\) (borrowing from
CT-VAR terminology) `drift' matrix that includes all lagged effects of
\(K\times1\) column vectors \(X_t\) to \(X_{t-1}\), \(K\) being the
number of observed items (symptoms). In this section the focus is on the
\(2K\times2K\) covariance matrix where two subsequent measurement time
points are observed. All matrices used are real-valued.

First, the covariance matrix (assumed stationary over time) is

\[
\begin{align*}
\text{Cov}(X_t) &= E[X_tX_t^T] \\
&= E[(AX_{t-1}+\Gamma_t)(AX_{t-1}+\Gamma_t)^T] \\
&= E[AX_{t-1}X_{t-1}^TA^T] + \underbrace{E[\Gamma \Gamma^T]}_{=: \Psi} \\
&= AE[X_{t-1}X_{t-1}^T]A^T + \Psi \\
&= \Sigma_t=:\Sigma_{VAR(1)}
\end{align*}
\] where stationarity poses that \(\Sigma\) is not dependent of time and
so the covariance of VAR(1) is denoted as above from hereon.

The vectorized covariance matrix can be solved to equal\\
\[
\begin{align*}
\text{vec}(E[X_tX_t^T])&=\text{vec}(\Sigma_{VAR(1)}) \\
&=\text{vec}(AE[X_{t-1}X_{t-1}^T]A^T + \Psi) \\
&=\text{vec}(AE[X_{t-1}X_{t-1}^T]A^T) + \text{vec}(\Psi) \\
&=\text{vec}(A\otimes A)E[X_{t-1}X_{t-1}^T]+\text{vec}(\Psi) \\
\Longrightarrow\\
\text{vec}(E[X_tX_t^T])&=\text{vec}(A\otimes A)\text{vec}(E[X_{t-1}X_{t-1}^T])+\text{vec}(\Psi)&\Rightarrow\\
\text{vec}(E[X_tX_t^T])&=\text{vec}(A\otimes A)\text{vec}(E[X_{t-1}X_{t-1}^T])+\text{vec}(\Psi)&\Rightarrow\\
I&=\text{vec}(A\otimes A)+\text{vec}(\Psi)\text{vec}(E[X_tX_t^T])^{-1}&\Rightarrow\\
I-\text{vec}(A\otimes A) &= \text{vec}(\Psi)\text{vec}(E[X_tX_t^T])^{-1}&\Rightarrow\\
\text{vec}(E[X_tX_t^T])&= (I-A \otimes A)^{-1} \text{vec}(\Psi)
\end{align*}
\]

Where vec is the vectorization operator and \(\otimes\) is the Kronecker
prodcut. In the above the mixed Kronecker matrix vector product is used.
We will equate the VAR(1) posed \(\Sigma_{VAR(1)}\) to s-LMI imposed
covariance further down below.

Second, VAR(1) poses that observations at the time points
\(X_t, X_{t-1}\) have covariance \[
\begin{align*}
\text{Cov}(X_t,X_{t-1})&=
E[X_tX_{t-1}^T]-E[X_t]E[X_{t-1}]\\&=
E[(AX_{t-1}+\Gamma_t)X_{t-1}^T]\\&=
E[AX_{t-1}X_{t-1}^T]+E[\Gamma_tX_{t-1}^T]\\&=
A\Sigma_{t-1}+E[\Gamma_tX_{t-1}^T]
\end{align*}
\] Independent errors means that
\(E[\Gamma_tX_{t-1}^T]=Cov(\Gamma_t,X_{t-1})=0\) leading to \[
\begin{align*}
\text{Cov}(X_t,X_{t-1})&= A\Sigma_{t-1}\\
\text{Cov}(X_{t-1},X_t)&= \Sigma_{t-1}^T A^T \\
\end{align*}
\] Where the two covariances above must be the same - i.e., the
covariance matrix is symmetric. This means that every VAR(1) process
implies that covariance of observations from two subsequent time points
\(t, t-1\) is \[
Cov((X_{t-1},X_{t}),(X_{t-1},X_{t})) = 
\begin{pmatrix} 
  \Sigma_{t-1} & A\Sigma_{t-1} \\
  \Sigma_{t-1}^TA^T & \Sigma_t
\end{pmatrix} \] Where the above covariance matrix is the
\(2K\times 2K\) covariance matrix of the observed data from the two time
points. In addition, stationarity directly implies
\(\Sigma_t=\Sigma_{t-1}\). (For now, notation with sub-index \emph{i}
will be kept for clarity as it is. s-LMI is not necessarily stationary,
so confusion might be avoided.)

\hypertarget{s-lmi-covariance-structure-at-two-subsequent-time-points}{%
\paragraph{s-LMI covariance structure at two subsequent time
points}\label{s-lmi-covariance-structure-at-two-subsequent-time-points}}

s-LMI with 1 common factor decomposes \(\Sigma_{t-1}\) into following
\[\Sigma_{t-1}=\Lambda\Lambda^T+\Omega_{t-1}\]where by definition of
s-LMI \(\Omega_{t-1}\) is diagonal and and \(\Lambda\) is a \(K\times1\)
column vector of factor loadings constant over time. We also need the
covariance of the common factor at both time points. Let \(\delta\) be
the latent regression coefficient which links the common factor to
itself at a previous time point such that
\(\eta_t=\delta\eta_{t-1}+\psi_t\), where \(\psi_t\) is independent
random term (`innovation', `error', `disturbance') with \(E[\psi_t]=0\).
Assuming standardized common factor such that
\(E[\eta_{t-1}]=0,\:Var(\eta_{t-1})=1\) covariance of the common factor
at two subsequent time points is

\[
\begin{align*} \text{Cov}(\eta_{t-1},\eta_t)= E[\eta_{t-1}\eta_t]-E[\eta_{t-1}]E[\eta_t]&=\\ E[\eta_{t-1}(\delta\eta_{t-1}+\psi_t)]&=\\ E[\delta\eta_{t-1}^2+\eta_{t-1}\psi_t]&=\\ \delta Var(\eta_{t-1})&= \delta \end{align*}
\] Now lets look at the \(2K\times 2K\) covariance matrix from the
perspective of strict LMI. A s-LMI model imposes that \[
\text{Cov}((X_{t},X_{t-1}),(X_{t},X_{t-1}))= \begin{pmatrix}    \Lambda & 0
\\   0 & \Lambda \end{pmatrix}  \begin{pmatrix}   1 & \delta 
\\   \delta & \delta+Var(\psi_t) \end{pmatrix} \begin{pmatrix}    \Lambda^T & 0
\\   0 & \Lambda^T \end{pmatrix} + \begin{pmatrix}   \Omega_{t-1} & \Omega_{across} 
\\   \Omega_{across}^T & \Omega_t \end{pmatrix} 
\]

where

\[\begin{pmatrix}    \Lambda & 0\\   0 & \Lambda \end{pmatrix}\] is a
block diagonal matrix that sandwiches the \(2\times2\) covariance matrix
of the common factor at both time points.

\[
\begin{align*}&\begin{pmatrix}    \Lambda & 0\\   0 & \Lambda \end{pmatrix} \begin{pmatrix}   1 & \delta \\   \delta & \delta+Var(\psi_t) \end{pmatrix} \begin{pmatrix}    \Lambda^T & 0\\   0 & \Lambda^T \end{pmatrix}+ \begin{pmatrix}   \Omega_{t-1} & \Omega_{across} \\   \Omega_{across}^T & \Omega_t \end{pmatrix} =\\
&\begin{pmatrix}    \Lambda & 0\\   0 & \Lambda \end{pmatrix} \begin{pmatrix}    \Lambda^T & \delta\Lambda^T\\   \Lambda^T\delta & (\delta+Var(\psi_t)\Lambda^T \end{pmatrix} + 
\begin{pmatrix}   
\Omega_{t-1} & \Omega_{across} \\   
\Omega_{across}^T & \Omega_t \end{pmatrix}=\\
& \begin{pmatrix}    
\Lambda\Lambda^T + \Omega_{t-1}& \Lambda\Lambda^T\delta + \Omega_{across}\\   
\Lambda\Lambda^T\delta + \Omega_{across}^T & \Lambda\Lambda^T(\delta+Var(\psi_t)) + \Omega_t 
\end{pmatrix} 
\end{align*}
\] From the above we see that the strict LMI can only be compatible with
any process with stationary covariance, if
\(\delta+Var(\psi_t)=1\Rightarrow1-\delta=Var(\psi_t)\) (assuming
\(\Lambda\) is non-zero). (When fitting a s-LMI model this is allowed.)
We also see that s-LMI is compatible with non-stationary processes where
the covariance is proportional to \(\delta+Var(\psi_t)\) aligning with
previous theoretical {[}Note: insert Tom's analysis{]} analysis where
covariance increased over time in a LMI preserving model.

A brief note on notation: We'll be using simply \(\Omega\) for the s-LMI
residual covariance, since residual covariance is assumed invariant over
time \(\Omega_{t-1}=\Omega_{t}=\Omega_{t+1}=…=\Omega\) .

Using the above auxiliary results we can move to analyse the null
hypothesis (hypotheses) of no difference between VAR(1) and s-LMI.

\hypertarget{working-null-hypothesis-1-if-covariance-matrix-generated-by-a-true-var1-model-at-some-measurement-time-point-is-perfectly-explained-by-a-common-factor-model-then-s-lmi-model-fits-perfectly.}{%
\paragraph{Working null hypothesis (1): If covariance matrix generated
by a true VAR(1) model at some (measurement) time point is perfectly
explained by a common factor model, then s-LMI model fits
perfectly.}\label{working-null-hypothesis-1-if-covariance-matrix-generated-by-a-true-var1-model-at-some-measurement-time-point-is-perfectly-explained-by-a-common-factor-model-then-s-lmi-model-fits-perfectly.}}

Considering only the subset of VAR(1) processes which create a
covariance matrix that can be perfectly explained by a common factor
model is done because if true it tells us how (if at all) VAR(1) can
deviate from s-LMI. Understandably, if any VAR(1) model creates
covariance structure incompatible with s-LMI model at some time point,
then deviation must occur (although the extent to which this occurs is
not clear at this point).

If the VAR(1) generated \(2K\times2K\) matrix cannot be explained by the
strict LMI model, this seems likely to be because the off diagonal
blocks of covariance matrices across time points are non-compatible with
the respective s-LMI model imposed across time covariance. Combined with
the restriction on the within time point covariance, this might lead to
contradictions.

This gives us the following null hypothesis (1) equations (from the
\(2K\times2K\) matrices imposed by VAR(1) and s-LMI)

\[
\begin{align*}
(I-A \otimes A)^{-1} \text{vec}(\Psi) &= \text{vec}(\Lambda \Lambda^T + \Omega)
\end{align*}
\]

and

\[
\begin{align*}
&A\Sigma_{t-1}=\Lambda\Lambda^T\delta+\Omega_{across}
\end{align*}
\]

both of which must be true for the null hypothesis (1) to hold. Assuming
that the null hypothesis (1) is true, further analysis of the respective
equations show

\[
\begin{aligned}
A\Sigma_{t-1} &= \Lambda\Lambda^T\delta+\Omega_{cross} \Leftrightarrow\\
A &= \Lambda\Lambda^T\delta \Sigma_{t-1}^{-1} + \Omega_{cross} \Sigma_{t-1}^{-1} \Leftrightarrow\\
A + \delta \Omega \Sigma_{t-1}^{-1} &= \Lambda\Lambda^T\delta \Sigma_{t-1}^{-1} + \delta \Omega \Sigma_{t-1}^{-1} + \Omega_{cross} \Sigma_{t-1}^{-1} \Leftrightarrow\\
A + \delta \Omega \Sigma_{t-1}^{-1} &= \delta \underbrace{(\Lambda\Lambda^T + \Omega)}_{=\Sigma_{t-1} \text{ by assumption}} \Sigma_{t-1}^{-1} + \Omega_{cross} \Sigma_{t-1}^{-1} \Leftrightarrow\\
A &= \delta I + (\Omega_{cross} - \delta \Omega)\Sigma_{t-1}^{-1}.
\end{aligned}
\]

From the above equation we see that \(A\) must be symmetric under the
null hypothesis (1) because both \(\Sigma,\Sigma_{across}\) are
covariance matrices (or a precision matrix) and otherwise only
re-scaling with a scalar, multiplying by a diagonal matrix (since
diagonal matrix commutes with all matrices) is done for symmetric
matrices.

Combining, we get a system of (matrix) equations

\[
\begin{cases}
&(I-A \otimes A)^{-1} \text{vec}(\Psi) &=& \text{vec}(\Lambda \Lambda^T + \Omega)\\
&A &=& \delta I + (\Omega_{cross} - \delta \Omega)\Sigma_{t-1}^{-1}
\end{cases}
\]

substituting \(A\) into the upper equation

\[
\begin{align*}
(I-(\delta I + (\Omega_{cross} - \delta \Omega)\Sigma_{t-1}^{-1}) \otimes (\delta I + (\Omega_{cross} - \delta \Omega)\Sigma_{t-1}^{-1}))^{-1} \text{vec}(\Psi) &= 
\\  (I-(\delta I + (\Omega_{cross} - \delta \Omega)\Sigma_{t-1}^{-1}) \otimes\delta I + (\delta I + (\Omega_{cross} - \delta \Omega)\Sigma_{t-1}^{-1})\otimes ((\Omega_{cross} - \delta \Omega)\Sigma_{t-1}^{-1}))^{-1} \text{vec}(\Psi)&=
\\
(I-\delta I\otimes \delta I + (\Omega_{cross} - \delta \Omega)\Sigma_{t-1}^{-1} \otimes\delta I + \delta I\otimes((\Omega_{cross} - \delta \Omega)\Sigma_{t-1}^{-1})+  
\\ 
(\Omega_{cross} - \delta \Omega)\Sigma_{t-1}^{-1}\otimes (\Omega_{cross} - \delta \Omega)\Sigma_{t-1}^{-1})^{-1} \text{vec}(\Psi)&=
\\
(I-\delta^2I+2(\delta I\otimes(\Omega_{cross} - \delta \Omega)\Sigma_{t-1}^{-1})
+(\Omega_{cross} - \delta \Omega)\Sigma_{t-1}^{-1}\otimes (\Omega_{cross} - \delta \Omega)\Sigma_{t-1}^{-1})^{-1} \text{vec}(\Psi)&=
\\
\\
\text{where is this goin?}
\\
\\
(I-\delta^2I+2(\delta I\otimes((A-\delta I)))+(A-\delta I)\otimes (A-\delta I))^{-1} \text{vec}(\Psi)&=
\\
(I-\delta^2I+2(\delta I\otimes(A-\delta I))+A\otimes A-A\otimes\delta I-\delta I\otimes A+\delta^2I)^{-1} \text{vec}(\Psi)&=
\\
(I+\delta I\otimes(A-\delta I)+\delta I\otimes(A-\delta I)+A\otimes A-A\otimes\delta I-\delta I\otimes A)^{-1} \text{vec}(\Psi)&=
\\
(I+\delta I\otimes A-\delta^2I+\delta I\otimes A-\delta^2I+A\otimes A-A\otimes\delta I-\delta I\otimes A)^{-1} \text{vec}(\Psi)&=
\\
(I+\delta I\otimes A-\delta^2I+\delta I\otimes A-\delta^2I+A\otimes A-A\otimes\delta I-\delta I\otimes A)^{-1} \text{vec}(\Psi)&=
\\
&=\text{vec}(\Lambda \Lambda^T + \Omega)\\
\end{align*}
\]

\hypertarget{var1-covariance-compared-to-s-lmi-covariance-at-t-time-points.}{%
\paragraph{VAR(1) covariance compared to s-LMI covariance at T time
points.}\label{var1-covariance-compared-to-s-lmi-covariance-at-t-time-points.}}

As the scenario where two subsequent measurement points are observed is
possibly the most common one, we'll keep the main null hypothesis (1) as
the respective scenario. On the other hand, it is of interest to analyze
what happens when multiple subsequent time points are included. This is
perhaps less common in measurement invariance literature, but more
common in VAR literature.

We have the assumption (proven below) that as the distance in time
between two time points \(\Delta t=(t_1-t_2)\to\infty\) samples from
VAR(1) model generated data at those two time points have 0 covariance.
This makes the \(2K\times2K\) perfectly explained by a measurement
invariance model, because the main diagonal covariance matrices
\(\Sigma_{t-1}=\Sigma_t\) are perfectly explained by a common factor
model by assumption, and the off diagonal matrices are 0, which is
allowed in a strict LMI model (no cross-covariance between the observed
time points and 0 regression coefficient for the latent variable). This
is a fairly simple, although less practically meaningful case. Scenarios
when 0 across time points covariance are not as often observed arguably.
This also does prove that a VAR(1) process could in some sense be the
true data generating process even if no across time point covariance is
observed (no lagged effects are estimated) if one were to claim that
\(\Delta t\) is very small: We're just not observing time points close
enough to each other to see the VAR(1).

Nevertheless - considering that the above scenario is not a typical one
in psychopathology research - we can attempt to generalize the above
results concerning the \(2K\times2K\) matrix to an \(TK\times TK\)
matrix where we have \(T\) measurement of \(K\) symptoms over occasions
at constant time intervals. Brief notes are made as we move on and a
summary at the end.

Here we change the notation a little and use an arbitrary time point 1
as the first measurement time point, increasing \({1, 2,...,T}\). The
\(TK\times TK\) matrix is

\[
\begin{array}
  \\\Sigma_1&...&\Sigma_{1,T}
  \\ \vdots&\ddots&\vdots
  \\ \Sigma_{1,T}^T&...&\Sigma_T
\end{array}
\]

Proceeding again from VAR(1) to s-LMI and then equating between the
models. Let \(^{(T)}\) denote the matrix raised to power of \(T\). The
\(\Sigma_{1,T}\) for VAR(1) is

\[
\Sigma_{1,T}=\text{Cov}(X_1, X_{T}) = A^{(T)}\Sigma_{VAR(1)}
\]

On a brief note we can further decompose the above equation, using the
power method of eigenvalues, into

\[
PD^{(T)}P^{-1}\Sigma_{VAR(1)}
\]

where \(D\) is a diagonal matrix of eigenvalues of \(A\), \(P\) is and
orthonormal matrix of eigenvectors of \(A\). From this decomposition we
directly see the above mentioned property that as distance in time
between time points increases, \(D\) is raised to a larger power and
decreases eventually to the zero matrix. This is because eigenvalues of
(stationary) \(A\) are less than one, meaning that all diagonal elements
of \(\text{diag}(D):|d_{ii}|<1\) as well. This means that a true data
generating stationary VAR(1) model should produce across time point
covariances that reduce to 0 as distance in time between any two time
points increases. Such observations might be sparse in psychopathology
literature - i.e., it is accepted in the literature that even if any
autoregressive model would be the true data generating model, it would
unlikely be stationary over lengthier time periods (e.g., years).

For s-LMI, respectively, we'll use
\(\delta_2, \delta_3, ...,\delta_{t+1},..., \delta_T\) to denote the
regression coefficient between subsequent time points. Also, the
\(\Omega_{across}\) needs to be generalized to include residual
covariances between any two time points so that \(\Omega_{1,T}\) is the
residual covariance between time point 1 (and \(\Omega\) is a within
time point residual covariance) and time point \(T\).

The common factor regression equation also changes to
\(\eta_{t+1}=\delta_{t+1}\eta_{t}+\psi_{t+1}\) leading to

\[
\begin{align*}
\text{Cov}(\eta_t,\eta_{t+2})&=E[\eta_{t}\eta_{t+2}]\\
&=E[\eta_{t}  (\delta_{t+2}\eta_{t+1}+\psi_{t+2})  ]\\
&=E[\eta_{t}  (\delta_{t+2}(\delta_{t+1}\eta_{t}+\psi_{t+1})+\psi_{t+2})  ]\\
&=E[\eta_{t}  (\delta_{t+2}\delta_{t+1}\eta_{t}+\delta_{t+2}\psi_{t+1}+\psi_{t+2})  ]\\
&=E[\delta_{t+2}\delta_{t+1}\eta_{t}\eta_{t}]+E[\delta_{t+2}\psi_{t+1}\eta_{t}]+E[\psi_{t+2}\eta_{t}]\\
&=\delta_{t+2}\delta_{t+1}
\end{align*}
\]

and further

\[
\text{Cov}(\eta_t,\eta_{t+\Delta t})=\prod_{t}^{t+\Delta t}\delta_t
\]

Luckily, we observe that the VAR(1) imposed covariance between any two
subsequent time points is the same. This means that

\hypertarget{the-above-could-be-proven-by-induction-but-for-now-omitted.}{%
\subparagraph{(The above could be proven by induction but for now
omitted.)}\label{the-above-could-be-proven-by-induction-but-for-now-omitted.}}

From the previous result for the \(2K\times2K\) matrix we can then
generalize to the \(TK\times TK\) matrix

\[
\begin{array}
\\\Lambda\Lambda^T + \Omega & ... &  \Lambda\Lambda^T \prod_{t=2}^{T}\delta_t + \Omega_{1,T}
\\ \vdots&\ddots&\vdots
\\ \Lambda\Lambda^T \prod_{t=2}^{T}\delta_t + \Omega_{1,T}^T&...&\Lambda\Lambda^T(\delta_T+Var(\psi_t)) + \Omega
\end{array}
\]

Now we can obtain the more general, auxiliary null hypothesis equation
(A1) equations relating the across time point covariances and within
time point covariances for any lag

\[
\begin{align*}
\Sigma_{1,T}&=\\
A^{(T)}\Sigma&=\Lambda\Lambda^T \prod_{t=2}^{T}\delta_t + \Omega_{1,T}&&\Rightarrow\\
A^{(T)}&=(\Lambda\Lambda^T \prod_{t=2}^{T}\delta_t + \Omega_{1,T})\Sigma^{-1}
\end{align*}
\]

which is symmetric, as was the case for the null hypothesis equation
(1). (The above is true for all \(t\in1,..,T\).) This is because

\hypertarget{summary-and-implications}{%
\subsubsection{Summary and
implications}\label{summary-and-implications}}

\begin{itemize}
\item
  A direct observation from the null hyopothesis (1) and auxiliary null
  hypothesis (A1) equations shown above is that \(A\) must always be
  symmetric for the common factor s-LMI model to be compatible with
  stationary VAR(1) generated data.
\item
  s-LMI is compatible with the idea that VAR(1) poses that across time
  point covariance approaches 0 as the distance between two time points
  increases. This scenario might not be frequently observed however,
  suggesting that - at least some part of - psychopathology processes
  cannot be understood as vector autoregressive processes. This is true
  for any stationary process to my knowledge, as the requirement is that
  the process will necessarily remain bounded to some extent to its
  `state'. In fact, it might be reasonable to integrate both viewpoints
  as is done in trait-state models.
\end{itemize}

\hypertarget{below-is-outdated}{%
\subparagraph{below is outdated!}\label{below-is-outdated}}

Few of specific scenarios that restrict the possible VAR(1) model set
are seen.

\begin{itemize}
\item
  We see that in the case where no across time idiosyncratic covariances
  exist (\(\Omega_{across} = 0\)), the right-hand-side becomes a
  diagonal matrix and hence \(A\) must also be diagonal.
\item
  We see that when \(\delta=0\), i.e.~there is no lagged effect on the
  common factor and \(\Omega_{across} = 0\), \(A\) is 0.
\item
  We see that when \(\delta=0\), and \(\Omega_{across}\ne0\),
  \(A=\Omega_{across}\Sigma_{t-1}^{-1}\:\Longrightarrow\: A\Sigma_{t-1}=\Omega_{across}\)
  means that there is no cross-symptom covariance across time.
\end{itemize}

All of these cases are very specific and arguably would not be
consistent with symptom-network theory. That is, 0 cross-lagged effects,
or 0 lagged effects at all are not consistent with symptom-network
theory. These are possibly somewhat rarely observed.

Perhaps the more general result is that since the right-hand-side is
always symmetric (since the diagonal matrix \(\deltaI\) is symmetric,
and a covariance matrix multiplied by a diagonal matrix is also
symmetric, as is the sum of two symmetric square matrices) then \(A\)
must also be symmetric in all cases for the null hypothesis (1) to hold.
This constrains set of s-LMI compatible VAR(1) models generally.

\hypertarget{implications}{%
\subsubsection{Implications}\label{implications}}

\begin{itemize}
\tightlist
\item
  We see that \(A\) must be symmetric since the rhs of the above only
  contains symmetric matrices. This places a constraint on \(A\), and
  implies that s-LMI fit to data generated from VAR(1) process varies as
  a function of asymmetry. It would then make sense to focus on
  asymmetry of \(A\) when creating simulated observations from symptom
  network and common factor distributions \(F_{SN},F_{CF}\)
\item
  Measures of asymmetry such as \(s\)
  \begin{equation}\protect\hypertarget{eq-asymmetry}{}{
  s \equiv (|A_{sym}|-|A_{anti}|)/(|A_{sym}|+|A_{anti}|)
  }\label{eq-asymmetry}\end{equation} where \(A\) is decomposed into its
  symmetric and asymmetric parts can be used also. The metric above is
  shown at:
  \href{https://math.stackexchange.com/questions/2048817/metric-for-how-symmetric-a-matrix-is}{stack
  exchange}. Asymmetry of \(A\) could further be approached analytically
  for example by decomposing \(A\) into its symmetric and asymmetric
  parts, and/or through simulations where asymmetry of \(A\) is varied
  by producing random matrices with some logic with how asymmetric is
  produced in \(A\).
\item
  Perhaps one possibility is something like Chi-squared testing with
  vectorized (non-redundant) squared elements of
  \(A_{lower-tri}^T-A_{upper-tri}\).
\item
  At this point it still is not evident that VAR(1) can generate a
  covariance matrix perfectly decomposable to a common factor model(?).
  Unless we see it directly from (1), which might be the case since
  \(\delta, \Omega\) are freely estimated and so \(A\) maybe exists.
\item
  Also it is not clear what happens when sampling from a population at
  different time points so that subjects are sampled at different time
  lags.
\item
  Further analysis could extend to lags, with the probable result that
  powers of \(A\) multiply \(\Sigma\) to obtain across time covariance
  matrices for higher order lags.
\item
  Stationarity is also not necessarily a condition which should be
  imposed, which can be discussed further.
\item
  VAR(1) is a special case of CT-VAR. The respective transformations
  from CT-VAR to VAR(1) are available.
\end{itemize}

\hypertarget{section}{%
\paragraph{}\label{section}}



\end{document}
